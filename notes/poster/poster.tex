\documentclass[unknownkeysallowed,final]{beamer}
\usetheme{RJH} %RJH}
\usepackage[orientation=landscape,size=a2,scale=1.4,debug]{beamerposter}
\usepackage[absolute,overlay]{textpos}
\setlength{\TPHorizModule}{1cm}
\setlength{\TPVertModule}{1cm}

\usepackage{amsmath,amsthm, amssymb, latexsym}
\usepackage{multirow}
\usepackage{rotating}

\usepackage{array,booktabs,tabularx}
\newcolumntype{Z}{>{\centering\arraybackslash}X} % centered tabularx columns
\newcommand{\pphantom}{\textcolor{ta3aluminium}} % phantom introduces a vertical space in p formatted table columns??!!
\newcommand{\vectornorm}[1]{\left|\left|#1\right|\right|}

% Poster size can be: 36" wide by 44" tall

\title{\normalsize{Real-Time Activity Classification by Matched Filtering using Body-Worn Accelerometers}}
\author{\small{Craig~Euler craig.euler@gmail.com, C.T.~Lin ct.lin@csun.edu,\\Bryan~Juarez bryan.juarez.974@my.csun.edu, Melissa~Flores melissa.flores.455@my.csun.edu\\College of Engineering and Computer Science\\California State University, Northridge}}

\footer{}
\date{}

\begin{document}
\tiny{}
\begin{frame}{} 

\begin{textblock}{19.5}(1, 7)

\begin{block}{\small{Introduction}}



Various methods for activity classification using body-worn sensors exist.
Custom Decision Tree, Automatically Generated Decision Tree, and Artificial Neural Networks \cite{parkka_ermes_korpipaa_mantyjarvi_peltola_korhonen_2006} have been explored as well as other frequency based classification methods \cite{sharma_purwar_lee_lee_chung_2008}.
%The use of the matched-filter method for classification has been studied in the past \cite{giannakis_tsatsanis_1990} and we seek to adopt this method for our use to identify an individual\textquotesingle s activity for the purpose of real-time processing on a low-powered micro-controller.
The methods used for our algorithms are designed for computational efficiency to accommodate the resource limitations of a microcontroller. In this paper, we use the MSP432 microcontroller for our timing results.

%The algorithms described in this paper are used in two main modules: The activity classification module for real-time processing and the training module to train the device to identify the user\textquotesingle s activity.

The training module does not need to be performed in real time.
%This algorithm is the most demanding on computational resources and is only executed when training or re-training the unit to identify the user\textquotesingle s activity, which means that this can be performed on an external device such as a smart phone or tablet.
The purpose of the training module is to create the reference signals to be used on the microcontroller for the real-time activity classification processing.

The real-time activity classification module is intended to be performed on a low-powered microcontroller.
%This module reads in the references stored on the unit to be used as templates that the matched-filter will use for determining the individual\textquotesingle s activity.
Both of these modules use the same principles and hence the same algorithm kernels.



% Boltzmann equation
\begin{equation*}
\label{ES-BGK}
\partial_{t} f(t, \vec{x}, \vec{u}) + \vec{v} \cdot \vec{\nabla}_{x} f(t, \vec{x}, \vec{v}) = \nu (f_{0} - f)
\end{equation*}
% use \ref{ES-BGK} or \eqref{ES-BGK} to get the reference for this equation

Where $f_{0}$ is the Maxwellian equilibrium distribution

% f0
\begin{equation*}
\label{f0}
f_{0} = f_{M} = \frac{\rho(t, \vec{x})}{\sqrt{(2 \pi RT(t, \vec{x}))^{3}}} \exp \left( -\frac{( \vec{v}- \vec{\bar{u}}(t, \vec{x}))^{2}}{2RT} \right)
\end{equation*}
% use \ref{f0} or \eqref{f0} to get the reference for this equation

\end{block}

\begin{block}{\small{Reduction to One Dimension}}
We assume that there is no mass transfer in the $y$ and $z$ directions (examples: gas between two parallel plates or in an infinitely wide pipe). In this case, the BGK equation reduces to

% First PDE evolution equation for f1 and f2
\begin{equation*}
\label{bgk1d1}
\partial_{t} f^{1,2}(t,x,v) + v \partial_{x} f^{1,2}(t,x,v) = Q^{1,2}(f^{1,2},v)
\lefteqn{\hspace{20mm}(1)}
\end{equation*}

%Where

\begin{equation*}
\label{Q1}
Q^{1}(f^{1},v) = \nu \left( \frac{\rho(t, x)}{\sqrt{2 \pi RT(t, x)}} \exp \left( - \frac{(v - \bar{u})^{2}}{2RT} \right) - f^{1}(t, x, v) \right)
\end{equation*}

\begin{equation*}
\label{Q2}
Q^{2}(f^{2},v) = \nu \left( \rho(t, x) \sqrt{ \frac{RT(t, x)}{2 \pi}} \exp \left( - \frac{(v - \bar{u})^{2}}{2RT} \right) - f^{2}(t, x, v) \right)
\end{equation*}

%and

% Density in 1D
%\begin{equation*}
%\label{density_1d}
%\rho(t,x) = \int_{-\infty}^{\infty} f^{1}(t, x, v) dv_{1}
%\end{equation*}
% use \ref{density_1d} or \eqref{density_1d} to get the reference for this equation

% Momentum in 1D
%\begin{equation*}
%\label{momentum_1d}
%\bar{u}(t,x) = \frac{1}{\rho(t,x)}\int_{ -\infty}^{\infty} v_{1} f^{1}(t, x, v_{1}) dv_{1}
%\end{equation*}
% use \ref{momentum_1d} or \eqref{momentum_1d} to get the reference for this equation

% Temperature in 1D
%\begin{equation*}
%\label{temperature_1d}
%T(t,x) = \frac{1}{3R \rho(t,x)} \int_{-\infty}^{\infty} \left[ (v_{1} - \bar{u})^{2} f^{1}(t, x, v_{1}) + 2 f^{2}(t, x, v_{1}) \right]dv_{1}
%\end{equation*}
% use \ref{temperature_1d} or \eqref{temperature_1d} to get the reference for this equation

Where $f^{1} = \int f \, dv \, dw $ can be a reduced molecular distribution function and $f^{2} = \int v^{2} f \, dv \, dw $.
\end{block}

\begin{block}{\small{DG Velocity Discretization}}
We let $V=[v_{L}, v_{R}]$ be an interval large enough so that the first moment integrals of the distribution function outside of it are negligible and let it be partitioned into subintervals $I_{i}=[v_{i-1/2},v_{1+1/2}]$. Let $\chi_{p}$ and $\omega_{p}$, $p=1,\ldots,s$ denote the nodes and weights of the Gaussian quadrature of order $P$ on the interval $[-1,1]$. On $[v_{i-1/2},v_{i+1/2}]$ we define

\begin{equation*}
\label{Chi}
\chi_{p,i} = \frac{v_{i+1/2} + v_{i-1/2}}{2} + \chi_{p}\frac{v_{i+1/2} - v_{i-1/2}}{2}.
\end{equation*}
% use \ref{Chi} or \eqref{Chi} to get the reference for this equation

Our basis functions $\varphi_{p,i}(v)$ are
\begin{equation*}
\varphi_{p,i}(v) = \prod\limits_{q=1,s \atop q \neq p} \frac{v-\chi_{q,i}}{\chi_{p,i} - \chi_{q,i}}.
\end{equation*}
\end{block}

\end{textblock}

\begin{textblock}{18}(21.8,4)

 \begin{block}{\small{}}
Notice that basis functions satisfy the orthogonality properties

\begin{equation*}
\int_{v_{i-1/2}}^{v_{i+1/2}} \phi_{p,i}(v) \phi_{q,i}(v) = \frac{\Delta v_{i}}{2} \omega_{p} \delta_{p,q}
\end{equation*}

and

\begin{equation*}
\int_{v_{i-1/2}}^{v_{i+1/2}} v \phi_{p,i}(v) \phi_{q,i}(v) = \frac{\Delta v_{i}}{2} \chi_{p,q} \omega_{p} \delta_{p,q},
\end{equation*}

where $\delta_{p,q}$ is the Kronecker delta function.

The discrete velocity approximations of $f^{1,2}(t,x,v)$ are defined by

\begin{equation*}
\label{eq0x}
f^{1,2}(t,x,v)=\sum_{p=1} ^{s} f_{p,i}^{1,2}(t,x) \varphi_{p,i}(v).
\end{equation*}
% use \ref{base} or \eqref{base} to get the reference for this equation
Substituting the discrete velocity approximations into (1) and multiplying by our basis functions and integrating through velocity, after simplification we arrive at

\begin{equation*}
\label{eqint}
\partial_{t} f_{q,i}^{1,2}(t,x) + \chi_{p,i} \partial_{x} f_{q,i}^{1,2}(t,x) = Q^{1,2}(f^{1,2},\chi_{p,i}).
\end{equation*}

Notice that these equations have the form of a first order symmetric hyperbolic system.

\end{block}

\begin{block}{\small{Clawpack}}
Clawpack is a software package developed by LeVeque et al [1]. It is designed to provide solutions to Hyperbolic PDEs in 1 and 2 dimensions and of first or second orders. Current implementations of Clawpack include Geoclaw for Tsunami modelling and other geophysical flows. In our implementation, we seek to extend Clawpack to use for the solution of the model Boltzmann equations.
\end{block}

\begin{block}{\small{One Dimensional Heat Transfer}}
Nitrogen gas at 300K between two infinitely long parallel plates. The plate on the left is at 0m and is heated to 1000K. The plate on the right is at 0.1m and is heated to 300K. At 0.03 seconds the solution is observed to be near steady state.\\[2mm]
Our first experiment is to study the relative error in the total mass. The number of cells in space ranged from 10-100 and the number of cells through velocity ranged from 6 to 16 with a scheme order of 10 for velocity approximation.
We observed by varying the number of velocity cells we can reduce the error to a factor of $10^{-10}$. It is observed that varying the number of cells in the spatial variable $x$ has little effect on the conservation of mass as seen in the table below.

\end{block}
\end{textblock}

\begin{textblock}{17.3}(41,4)

\begin{block}{\small{}}

\begin{center}
\begin{tabular}{ cc || c | c | c | c | c | c | c | c | c | r |}

& & \multicolumn{4}{| c |}{Number of velocity cells}\\
& & 6 & 8 & 12 & 16\\ \hline \hline
\multirow{10}{*}{\begin{sideways}{Number of spatial cells}\end{sideways}}
& 10 & 5.30E-07 & 1.83E-09 & 1.26E-09 & 2.16E-10\\ \cline{2-6}
& 20 & 5.33E-07 & 1.34E-09 & 7.81E-10 & 4.33E-10\\ \cline{2-6}
& 30 & 5.24E-07 & 1.44E-08 & 1.02E-08 & 9.34E-09\\ \cline{2-6}
& 40 & 5.33E-07 & 2.53E-09 & 7.89E-10 & 6.15E-10\\ \cline{2-6}
& 50 & 5.34E-07 & 2.12E-09 & 1.26E-09 & 3.06E-11\\ \cline{2-6}
& 60 & 5.54E-07 & 1.63E-08 & 1.93E-08 & 2.02E-08\\ \cline{2-6}
& 70 & 5.13E-07 & 2.25E-08 & 1.94E-08 & 1.93E-08\\ \cline{2-6}
& 80 & 5.34E-07 & 3.18E-09 & 2.73E-10 & 1.19E-09\\ \cline{2-6}
& 90 & 5.25E-07 & 1.24E-08 & 9.60E-09 & 9.79E-09\\ \cline{2-6}
& 100 & 5.33E-07 & 2.54E-09 &  5.10E-10 & 1.14E-10\\ \hline

\hline
\end{tabular}
\end{center}

The convergence of the L2 norm of the error in macroparameters with respect to the resolution in the spatial variable can be seen in the table below. The solutions are compared to solutions on 1000 cells. The velocity discretization is by a scheme of order 10 on 10 velocity cells.

\begin{center}
\begin{tabular}{ lc || c | c | c | c |}

& & \multicolumn{4}{| c |}{Number of spatial cells}\\
& & 50 & 100 & 150 & 200\\ \hline \hline
& Density & 0.03054 & 0.01737 & 0.01198 & 0.008234\\ \cline{2-6}
& Momentum & 17.06 & 8.383 & 5.368 & 3.842\\ \cline{2-6}
& Temperature & 0.03967 & 0.02408 & 0.01652 & 0.01062\\ \hline

\hline
\end{tabular}
\end{center}



\end{block}

\begin{block}{\small{Steady State Solution}}
\begin{center}
\includegraphics[scale=.75]{reference_pulse_32Hz_4sec_8downsample_2.png}
\end{center}
\end{block}

\begin{block}{\small{References}}
\bibliographystyle{plain}
\bibliography{reference}

\end{block}

\end{textblock}

\end{frame}
\end{document}
