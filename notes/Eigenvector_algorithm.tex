\documentclass[11pt]{amsart}
\usepackage{multirow}
\usepackage{rotating}
\usepackage{amsmath}

\title{\LaTeX}
\date{}

\def\R{\mathbb{R}}
\def\T{\mathbb{T}}

\begin{document}

\begin{center}
Eigen-vector and Eigen-value Calculation Algorithm
\end{center}

\section{Method}
Each matrix has a unique QR factorization (check on this). The method used for computing the eigen-* of a symmetric matrix is with a repeated computation of QR decompositions. A given matrix $A$ can be represented in its decomposted form $A = QR$. Then $Q'AQ = RQ$ ($Q' = Q^{-1}$). Representing $A_1 = RQ$ and iteratively repeating the process will eventually converge to a diagonal matrix whose diagonal represents the eigen-values and the product of the $Q_i$ terms on the right represents the respective eigen-vectors.  Proofs of these statements can be found in any standard text on or relating to numerical linear algebra.

The method used here for QR Factorization is with Householder reflections.

\section{QR Factorization}
The collision operator of the boltzmann equation is

\begin{equation}
\label{BoltzmannRHS}
\int_{\mathbb{R}^3} \int_0^{2 \pi} \int_0^{\frac{\pi}{2}} \left( f' f_1' - f f_1 \right) \sigma g sin(\theta) d \theta d \epsilon d u'
\end{equation}
%
Where $g$ is the relative velocity of colliding molecules and $\sigma = b/(sin(\theta) \frac{\partial \theta}{\partial b})$ is the differential cross section where $b$ is the radius of the molecule.
\begin{equation*}
f_m \int_{\mathbb{R}^3} \int_0^{2 \pi} \int_0^{\frac{\pi}{2}} f_{M1} \sigma g sin(\theta) d \theta d \epsilon d u' - f \int_{\mathbb{R}^3} \int_0^{2 \pi} \int_0^{\frac{\pi}{2}} f_1 \sigma g sin(\theta) d \theta d \epsilon d u'
\end{equation*}
%

\end{document}
